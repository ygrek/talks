\usepackage{fontspec}
\usepackage{graphicx,listings,verbatim,hyperref,color}
\usepackage{cite,enumerate,float,indentfirst}

\hypersetup{colorlinks=true}

\definecolor{dkgreen}{rgb}{0,0.6,0}
\definecolor{gray}{rgb}{0.5,0.5,0.5}
\definecolor{mauve}{rgb}{0.58,0,0.82}

\lstset{frame=tb,
  language=ml,
  aboveskip=3mm,
  belowskip=3mm,
  showstringspaces=false,
  columns=flexible,
  basicstyle={\tiny\ttfamily},
  numbers=none,
  numberstyle=\tiny\color{gray},
  keywordstyle=\color{blue},
  commentstyle=\color{dkgreen},
  stringstyle=\color{mauve},
  breaklines=true,
  breakatwhitespace=true
  tabsize=3,
  morekeywords={try,match,module,external},
  frame=single
}

\newcommand{\cmd}[1]{\texttt{{#1}}}
\newcommand{\run}[1]{\begin{tiny}\cmd{{#1}}\end{tiny}}

\title{OCaml 4.02}
\author{ygrek}
\institute{OCaml@Singapore}
\date{Nov 25, 2014}

\begin{document}

\begin{frame}
\titlepage
\end{frame}

\begin{frame}
\frametitle{Disclaimer}
This talk is a shameless ripoff and compilation of OCaml changelog, documentation, bugtracker and ICFP'14 slides.

All respect goes to original authors, all mistakes and misrepresentations are mine.
\end{frame}

\section{Timeline}

\begin{frame}
\frametitle{Past}
\begin{itemize}
\item \emph{3.12} (August 2010)
\begin{itemize}
\item polymorphic recursion
\item first-class modules
\item record fields punning
\end{itemize}
\item \emph{4.00} (July 2012)
\begin{itemize}
\item GADTs
\item \cmd{-bin-annot} typedtree dump
\end{itemize}
\item \emph{4.01} (April 2014)
\begin{itemize}
\item \cmd{-ppx} parsetree rewriters
\item type-level disambiguation of record labels
\item \cmd{(|>)} and \cmd{(@@)} in \cmd{Pervasives}
\end{itemize}
\end{itemize}
\end{frame}

\begin{frame}
\frametitle{Current}
\emph{4.02} (August 2014)
\begin{itemize}
\item attributes
\item module aliases
\item \cmd{-safe-string} read-only strings
\item match case for exceptions
\item generative functors
\item open types
\end{itemize}
\end{frame}

\begin{frame}
\frametitle{Future?}
\begin{itemize}
\item ephemerons
\item smarter inliner
\item modular implicits (aka scoped typeclasses)
\end{itemize}
\end{frame}

\section{4.02}

\begin{frame}
\frametitle{Attributes}
Standard way to put annotations on AST nodes from the source code. Those annotations can later be consumed
by \cmd{ppx} rewriters and the compiler itself. Possible (future) uses:
\begin{itemize}
\item anchors for code generation
\item per-scope enabling of certain compiler features (options, warnings)
\item deprecation markers (\cmd{[@@ocaml.deprecated]})
\item documentation comments
\end{itemize}
\end{frame}

\begin{frame}
\frametitle{Attributes (example)}
\lstinputlisting{attributes.ml}
\cmd{ocamlc -c -dparsetree attributes.ml} :
\lstinputlisting{attributes.parsetree}
\end{frame}

\begin{frame}[fragile]
\frametitle{Module aliases}
New signature item
\begin{lstlisting}
module A = B
\end{lstlisting}
which records and tracks the equality of module paths. The aliases are expanded only at the place
of usage, not at definition.

Benefits :
\begin{itemize}
\item solution to the long-standing issue with functors depending on  syntactic module paths equality
\item smaller compiled objects and interfaces
\item faster compilation
\item less link-time dependencies and smaller binaries (\cmd{-no-alias-deps})
\end{itemize}
\end{frame}

\begin{frame}
\frametitle{Read-only strings}
Introduces \cmd{type bytes} equal to \cmd{string} (preserving backward compatibility) unless
option \cmd{-safe-string} is used, in which case it is opaque.
Module \cmd{Bytes} in standard library can be used to create values of this type.
\par
The intended usage is: \cmd{Bytes} for writable buffers, \cmd{String} for read-only strings.
Option \cmd{-safe-string} is likely to become default in the future compiler release.
\end{frame}

\begin{frame}[fragile]
\frametitle{Read-only strings (cont.)}
\begin{lstlisting}
# #show_module Bytes;;
module Bytes :
  sig
    ...
    val make : int -> char -> bytes
    val init : int -> (int -> char) -> bytes
    val of_string : string -> bytes
    val to_string : bytes -> string
    val sub : bytes -> int -> int -> bytes
    val sub_string : bytes -> int -> int -> string
    ...
  end
# #show_module String;;
module String :
  sig
    external length : string -> int = "%string_length"
    external get : string -> int -> char = "%string_safe_get"
    external set : bytes -> int -> char -> unit = "%string_safe_set"
    ...
  end
\end{lstlisting}
\end{frame}

\begin{frame}
\frametitle{Match case for exceptions}
Well-known inconvenience for seasoned OCaml programmers: catching exceptions in tail recursive function.
\onslide<+->
\lstinputlisting{lines_bad.ml}
\onslide<+->
\lstinputlisting{lines_ugly.ml}
\onslide<+->
\lstinputlisting{lines_good.ml}
\end{frame}

\begin{frame}[fragile]
\frametitle{Generative functors}
Consider functor applied to the same module twice:
\begin{lstlisting}
module F (X : sig end) = struct type t end
module B1 = F(String)
module B2 = F(String)
let f (x:B1.t) : B2.t = x
(* val f : B1.t -> B2.t = <fun> *)
\end{lstlisting}
Should $B1.t$ be considered equal to $B2.t$ or not?
OCaml's functors are \emph{applicative} by default, which means that functor applications with equal parameters
yield modules with equal type components.

But sometimes \emph{generative} functors are desired, which generate new types for each application.
\begin{lstlisting}
module F (X : sig end) () = struct type t end
module B1 = F(String)()
module B2 = F(String)()
let f (x:B1.t) : B2.t = x
(* Error: This expression has type B1.t but an expression was expected of type B2.t *)
\end{lstlisting}
\end{frame}

\begin{frame}[fragile]
\frametitle{Open types}
OCaml already had one extensible type that could have constructors added after it's definition. This is
the type of exceptions : \cmd{exn}. It is special - one couldn't define new types with such behaviour. Not anymore:
\begin{lstlisting}
# type t = ..;;
type t = ..
# type t += A | B of int;;
type t += A | B of int
# B 2;;
- : t = B 2
# C 3.;;
Error: Unbound constructor C
# type t += C of int;;
type t += C of float
# C 3.;;
- : t = C 3.
\end{lstlisting}
Obviously, every pattern match on the value of open type should include wildcard case.
\end{frame}

\begin{frame}
\frametitle{FIN}
And all this goodness is just one command away. Do it now :
\begin{verbatim}
opam switch 4.02.1
\end{verbatim}
\end{frame}

\section{Community}

\begin{frame}
\frametitle{Community efforts}
\begin{itemize}
\item OPAM
\item merlin
\item github pull requests
\end{itemize}
\end{frame}

\begin{frame}
\frametitle{OPAM}
New OPAM 1.2 release adds several useful features.
\begin{itemize}
\item usable \cmd{opam pin}
\item \cmd{opam source}
\item \cmd{dev-repo} field in \emph{opam} files
\item \cmd{\{build\}} filter for \cmd{depends}
\end{itemize}
\end{frame}

\begin{frame}
\frametitle{merlin}
Code browsing tool for OCaml projects.

Integrates with Vim, Emacs, acme and Sublime Text, can be extended to support any capable editor.
Uses \emph{bin-annot} files produced by the compiler and handles partial builds.

Provides:
\begin{itemize}
\item type throwback for values under cursor
\item "Go to definition" across the whole project (and possibly for external libraries)
\item background build and on-the-fly error messages
\end{itemize}
\end{frame}

\begin{frame}
\frametitle{github}
Since some time ago there is an official mirror of OCaml compiler on github.

More importantly the pull requests against that repo are considered by Inria team.

Getting the patch into OCaml compiler had never been easier. Push it while it lasts!
\end{frame}

\section{References}

\begin{frame}
\frametitle{References}
\begin{enumerate}[1)]
\item \href{http://whitequark.org/blog/2014/04/16/a-guide-to-extension-points-in-ocaml/}{A Guide to Extension Points in OCaml}
\item \url{https://ocaml.org/meetings/ocaml/2014/}
\item \href{https://ocaml.org/meetings/ocaml/2014/OCaml2014-Leroy-slides.pdf}{The State of OCaml}
\item \href{https://ocaml.org/meetings/ocaml/2014/ocaml2014_2.pdf}{Ephemerons meet OCaml GC}
\item \url{https://github.com/ocaml/ocaml}
\end{enumerate}
\end{frame}

\end{document}
